\documentclass[onecolumn,preprintnumbers,amsmath,amssymb]{revtex4}
\newcommand{\gguide}{{\it Negative Correlation in Traffic Recovery}}
%Uncomment next line if AMS fonts required
\usepackage[dvips]{graphicx}% Include figure files
\usepackage{dcolumn}% Align table columns on decimal point
\usepackage{float}
\usepackage{bm}% bold math
\usepackage{color}


%\nofiles

\begin{document}

\title{Negative Correlation in Traffic Recovery}

\author{Yichao Zhang$^{1}$}
\email{yiczhang@cs.ucl.ac.uk}

\author{Xing Li$^{1}$}
\email{15000218895@163.com}


\author{Jihong Guan$^{1}$}
\email{jhguan@tongji.edu.cn}


\author{Shuigeng Zhou$^{2}$}
\email{sgzhou@fudan.edu.cn}

\affiliation{$^{1}$Department of Computer Science and Technology, Tongji University, 4800 Cao'an Road, Shanghai 201804, China}

\affiliation{$^{2}$Department of Computer Science and Engineering, Fudan University, Shanghai 200433, China \\ Shanghai Key Lab of Intelligent Information Processing, Fudan University, Shanghai 200433, China}

\begin{abstract}
Traffic congestion in traffic networks has so far been studied from the perspective of network planing.
The studies on how to efficiently mitigate the congested systems are relatively limited.
In this paper, we introduce an algorithm of recovering sequence planning, which is used to detect an efficient recovering sequence of congested links.
Following an efficient recovering sequence, we observe that the throughput of a congested traffic network can be recovered to a large extent by reconnecting few congested links. This highlights that the key role that the sequence plays in first-aid repairs in traffic systems.
We believe our algorithm is capable of promoting related studies on recovering congested traffic systems.
\end{abstract}

\pacs{87.23.Ge, % Negative Correlation
02.50.Le,       % Traffic Recovery
89.75.Fb        % Link Prediction
}

\maketitle

\section*{INTRODUCTION\protect}
Modern world depends greatly on the efficient operation of many critical infrastructures such as Internet, airlines, power grid, highways and so on \cite{TCIC}.
Such systems are neither regular nor random, which are typically modeled as networks. In the networks, nodes represent the basic units of a system and links stand for the interactions between the nodes \cite{TFWN}. 
In the studies of traffic networks, further promoting the capacity of the traffic systems is one of the ultimate aims. 
In this regard, most attention has been given to  traffic congestion problems in networks.
In the past decades, how to avoid  traffic congestions as  a key question has been extensively investigated.

Network transmission capacity is normally measured by the maximal amount of traffic flow that the network can handle without causing congestion \cite{TORS}. 
Knowledge on avoiding traffic congestion is relevant to the planning and controlling of  traffic systems.
Network planning concerns designing topological structures and deploying resources. 
Thus, corresponding studies are divide into two categories.
We briefly review the progresses in the respective categories here.

Providing a proper network structure for maximizing the capacity of traffic networks is a challenging task.
Boccaketti et al. suggest that in transportation networks, the global capacity depends on the load of each link.
That is to say the number of the passengers in the airports or the passengers in the subways. 
By removing a small fraction of nodes, the authours simulated the breakdown of an Internet router or transportation system.
They found that homogeneous network does not experience cascading failures due either to random breakdown or to intentional attacks. 
For the heterogeneous network, cascades triggered by the attack on a key node will lead to serious congestion problem.
Thus, homogeneous networks appear to be more resistant to cascading failures than the heterogeneous ones \cite{SBVV}.
Toroczkai et al. compared the congestion in gradient networks generated by either random or scale-free substrates, finding that the scale-free gradient networks are less prone to jamming \cite{ZTKE,ZTBE}.
Roger et al. claim that the system only behaves efficiently when the amount of handled packages  is small enough. 
The network collapses when the traffic exceeds a certain threshold. 
They found that the collection of models can be split into three groups according to how the network collapses if putting the topological effect aside. 
In the first group, agents deliver more packets as more packets are stranded.
Although their capability decreases, the network is not likely to collapse. 
In the second group, agents keep delivering the same number of packets regardless of their loads. 
The processing mechanism leads to a continuous phase transition. 
Finally, when agents deliver fewer packets as their loads increase, the transition to the congested phase is discontinuous and the network collapses in an inhomogeneous way giving rise to congestion nuclei \cite{RGA}.


In practice it is typically impossible to reform the topological structures of  existing traffic systems.
The results mentioned above are built on an assumption that all nodes or links are assigned with uniform resources. 
This simplified scenario allows researchers to study the pure contribution of the network topology, while resources are rarely distributed equally in realistic traffic networks.
In this regard Zhang et al. came up with a practical and effective approach to allocate the resources. 
They suggest that allocating more resources to the nodes handling higher volumes of traffic loads and less resources to the less-loaded nodes is helpful to keep the systems from congestion.
Their results show that their scheme enhances the transmission capacity of heterogeneous networks  by up to two orders of magnitude \cite{GQZ}.
Zhao et al. tested two resource allocations: 
(1) The capacity of delivery of each node is proportional to its degree; 
(2) The delivery capacity of each node is proportional to its betweenness. 
They have found that for the former allocation, random networks and scale-free networks are more tolerant to congestion than regular networks and Cayley trees. 
For the second allocation, the crowded degree is independent of the network topology and size, so they thus claim that the allocation may be useful for communication protocols \cite{LZY}.
% 好像就是普通的多叉树啊

Controlling traffic networks on the hand, concerns how to route traffic efficiently to mitigate congestion.
Daniele et al. tested a trade-off strategy between traffic based and topology based routing on scale-free networks.
They found that traffic control is useless in homogeneous networks but may useful in inhomogeneous networks \cite{DDL}. 
Gang Yan et al. proposed a routing strategy to improve the transportation efficiency on complex networks. 
Instead of using the routing strategy for shortest path, they proposed a generalized routing algorithm to detect the so-called ``efficient path''.
They take the nodes likely to be congested into consideration, which is in the shortest path.
Considering the nodes with larger degree are very susceptible to traffic congestion, they suggest that redistributing traffic load in central nodes to other non-central nodes is capable of efficiently refraining the traffic congestion.
Their simulation results show that the network capability in processing traffic is improved more than 10 times by routing with the efficient path \cite{GYT}.
Xiang et al. proposed a global dynamic routing strategy for network systems based on the queue lengths at nodes. 
Governed by the routing strategy, the traffic capacities of the corresponding networks were further improved \cite{GYT}. 
With time delay of updating node queue lengths and the corresponding paths, the system capacity remains constant, while the travel time for packets increases \cite{XLM}.
Tan Fei et al. proposed a routing algorithm that combines static structural properties and dynamic traffic conditions together, which balances the traffic between hubs and peripheral nodes more effectively \cite{FTY}.
Danila et al. proposed an algorithm to balance traffic on a network by minimizing the maximum node betweenness with a path as short as possible. 
Using the routing table, a network can sustain significantly higher traffic than adopting traditional shortest path routing \cite{BDY}.
Guan et al. investigated the traffic dynamics of a spatial scale-free network where the bandwidth of a link is proportional to its Euclidean distance.
They proposed a new routing strategy, considering of both Euclidean distance and betweenness centralities (BC) of links. 
They claim found that compared with the shortest distance path (SDP) strategy and the minimum betweenness centralities (MBC) of links strategy, their strategy under some specific settings can efficiently balance the traffic load and avoid excessive traveling distance which can improve the spatial network capacity and some metrics of transportation efficiency, such as average packets traveling time, average packets waiting time and system throughput, traffic load and so on \cite{XLMD}.

Compared with the potential costs of changing the structure of well-established networked systems, resource reallocation, proposals of clever routing criteria seem to be more practical and thus attract the most interest, while the controlling and designing approaches can not completely avoid congestion. 
Once the transportation system collapse, we need a solution to recover the capacity of the traffic system.. 
In this paper, we investigate the recovering sequence of congested links. 
Assuming that there are three congested links in a traffic network, while we just have one first-aid repair team in hand, where should we dispatch the team firstly?
In this case, there are six possible recovering sequences for us to choose, and it is obviously that different recovering sequence lead to different ends.
Clearly to find the best recovering sequence is a NP hard problem. 
Thus, we will provide a heuristic algorithm in this paper.
Our algorithms are based on local  structures, the time complexity is considerably low, which is applicable to the cases that the response time requirement is relatively high. 

% should be in the disscusion
%Topology structures of the transportation system can be easily obtained, and inspired by link prediction thoughts, we can predict the necessarily for two nodes to be connected. 
%Link prediction aims at finding missing links or predicting the emerging of new edges \cite{}.
%One big difference from link prediction, our goal is to to find the recovery sequences of the congested routes and these links are exist for sure.  
%In social networks prediction, if two nodes(person) get a higher number of common neighbors, then they are more likely to be friends, that is to say there will be a link
%between them in the future. 
%In the traffic system, sharing more neighbors means the two nodes shoulder a similar functions, so the connect of such edge may not enhance the transmission capacity efficiently, so the algorithms we will talk later will not recover such well connected links preferentially.  

Our contributions in this paper are three-fold. 
Firstly, our algorithms are proposed to recover the congested traffic systems, while
previous studies concentrate on the perspective of keeping the systems from congestions.
Thus, the motivation and scenario of our algorithms is clearly different from the existing algorithms. 
Secondly, for two classic network models and two realistic traffic networks, we propose and compare the performance of four algorithms.
They are RPA(reverse preferential attachment), RAPA(reverse add preferential attachment), RSA(reverse similarity attachment), RLP(reverse local path).
The algorithms are basically based on a reverse-similarity of nodes, which are rarely reported in the previous studies. 
Thirdly, through extensive traffic simulations, our results show that our algorithms are capable of detecting a more efficient recovering sequence than random recovery and that detected by a series of algorithms of link predictions. 



\section*{RELATED WORK\protect}
The simplest framework of link prediction methods is the similarity-based algorithm, where each pair of nodes x and y, is assigned a score $s_x$$_y$, which is directly defined as the similarity between x and y. Similarity-based algorithm are classified into three categories: local indices; global indices; quasi-local indices, which do not require global topological information but make use of more information than local indices\cite{SV}. Considered of computation complexity here in our experiment we use two classic algorithm: PA in local indices and LP in quasi-local indices and other transformed algorithms.

(1) Preferential Attachment Index (PA)\cite{WG}. The mechanism of preferential attachment can be used to generate evolving scale-free networks, where the probability that a new link is connected to the node x is proportional to k$_x$. A similar mechanism can also lead to scale-free networks without growth , where at each time step, an old link is removed and a new link is generated. The probability that this new link will connect x and y is proportional to k$_x$ × k$_y$. Motivated by this mechanism, the corresponding similarity index can be defined as
\begin{equation}
\centerline{
$S_x$$_y$=
k$_x$ * k$_y$
}
\end{equation}


(2) Local Path Index (LP)\cite{AC,LI}. Denote by A the adjacency matrix, where A$_x$$_y$ = 1 if x and y are directly connected, and A$_x$$_y$ = 0 otherwise. Obviously, (A$^2$)$_x$$_y$ is the number of common neighbors of nodes x and y, which is also equal to the number of different paths with length 2 connecting x and y. And if x and y are not directly connected , (A$^3$)$_x$$_y$ is equal to the number of different paths with length 3 connecting x and y. The information contained in A$^3$ can be used to break the degeneracy of the states, and thus we define a new measure as

\begin{equation}
\centerline{
$S$=
A$^2$ + $\varepsilon$A$^3$
}
\end{equation}
The score of $S_x$$_y$ correspond to the element is S. And in our experiment we set 
$\varepsilon$= 0.2.


(3) Three order Similarity Index (SA). This algorithm are proposed in this study, the purpose was to magnify the recovery accuracy of the traffic routes. $\Gamma$$^3$(x) means the set of first order neighbor second order neighbor and third order neighbor in all.
\begin{equation}
\centerline{
$S_x$$_y$=
$|$$\Gamma$$^3$(x) $\cap$ $\Gamma$$^3$(y)$|$
}
\end{equation}

\noindent {\bf The followed four algorithm are the transformer of the above three algorithm.}

(4) Reverse Preferential Attachment Index (RPA). simply put the numerator into denominator, which means nodes with less neighbors play a more important role.


\begin{equation}
S_{xy}=\frac{1}{k_x \times k_y}
\end{equation}

(5) Reverse Add Preferential Attachment Index (RAPA). One different from RPA is that at the denominator part we transfer multiply into add.


%$S_x$$_y$=$\frac{1}{k$_x$ * k$_y$}$
\begin{equation}
S_{xy}=\frac{1}{k_x + k_y}
\end{equation}

(6) Reverse Local Path Index (RLP). simply put the numerator into denominator

\begin{equation}
S=\frac{1}{A^2 + \varepsilon A^3}
\end{equation}
The score of $S_x$$_y$ correspond to the element is S. And in our experiment we set 
$\varepsilon$= 0.2.


(7) Reverse Three order Similarity Index (RSA). simply put the numerator into denominator
\begin{equation}
S_{xy}=\frac{1}{ |\Gamma^3(x) \cap \Gamma^3(y)|}
\end{equation}




\section*{DATASETS\protect}
To verify our methods, we have conducted varies experiments on three kinds of networks. The networks are as below:

\noindent {\bf BA network}

we select the Barabasi-Albert (BA)\cite{BA} model to generate the topological network, the degree distribution of which is a power law p(k) $\propto$ $k^{-3.0}$ where p(k) is the ratio of the nodes with degree k to the number of all nodes in the network. The model could represent the heterogeneous node degree of many real-world networks, including the Internet AS level topology, the logical topology of unstructured P2P distributed systems, and so on. In the simulation, we set the mean degree of the network k=4.

\noindent {\bf USAir}

We got USAir data from pajek datasets, The network is extracted from the USAir transportation system. The data contains 332 nodes and 2126 edges. Nodes in the network are individual cities and edges are airlines between these cities. Notice that this is a undirected network which means if you can buy a ticket from city A to city B then you can also buy a ticket from city B to city A. In some degree, USAir is a kind of BA network, some major City like New York and Los Angeles owns more airlines than small cities.

\noindent {\bf California road network}

The dataset was obtained from Stanford dataset. The original data contains 1,965,206 nodes and 2,766,607 edges. Considered the network is too large for link prediction and traffic simulation, so we split out some small part from the California road network. First we randomly choose a node and take it as the center node, then we use Level Traverse method to spread the network out and we can get different size of road network for our needs. For example, if we take nodeId= 100 as the center node, then we select the node 100’s neighbor and append them into the new network, then append the 100’s neighbor’s Neighbor, and the next steps are as before. In California road network, if we choose nodeId= 800120 as the center node and the spread level is 30 we can get a network with 1285 nodes and 1748 edges.

\noindent {\bf The basic topology statistics of the six networks are summarized in Table I.}

\begin{table}[tbp]
\centering  % 表居中
\begin{tabular}{lccccccc}  % {lccc} 表示各列元素对齐方式,left-l,right-r,center-c
\hline  % \hline 在此行下面画一横线
&$|V|$ &$|E|$ &D &C &$<k>$ &$<d>$ &H\\ 
\hline  BA256 &256 &508 &0.016 &0.075 &3.97 &3.49 &2.17 \\  
\hline  BA512 &512 &1020 &0.008 &0.052 &3.98 &3.68 &2.87 \\
\hline  BA1024 &1024 &2044 &0.004 &0.030 &3.99 &4.07 &2.78\\
\hline USAir &332 &2126 &0.039 &0.625 &12.81 &2.74 &3.46\\
\hline Cal Road 10 &121 &156 &0.021 &0.142 &2.58 &9.54 &1.13\\
\hline Cal Road 20 &409 &525 &0.006 &0.101 &2.57 &17.71 &1.17\\
\hline Cal Road 30 &1285 &1750 &0.002 &0.077 &2.72 &23.28 &1.16\\
\hline
\end{tabular}
\caption{statistics of the six networks.}
\end{table}

$|V|$ is the number of vertices and $|E|$ is the number of edges in a given network. D denotes the density of the network which is to calculate $\frac{2|E|}{|V-1|*|V|}$. A node’s clustering coefficient states that, if there are $k_i$ neighbors owned by a vertex $v_i$ and the maximal feasible edges among them could be $k_i*(k_i – 1)/2$, the local clustering coefficient for the vertex $v_i$ is to calculate $C_i$ = \cite{SW}, where $N_i$ stands for the number of neighbors of $v_i$. C denotes the average clustering coefficient. $|k|$ is the average degree of the network. Here, a shortest path is defined as a path connecting two unconnected nodes with least edges and thus the shortest path distance is the number of edges existed within the shortest path. $<d>$ denotes the average shortest path distance of the network. Degree heterogeneity is a statistical property, which quantitatively characterizes the fluctuation of the degree sequence of a network. This property was frequently mentioned, while its explicit definition was not manifested until a recent work\cite{PML}. To be consistent with previous studies, we denote the degree heterogeneity by H. H is defined as $\frac{2|E|}{|V-1|*|V|}$, where $<k>$ denotes the average degree\cite{KD}.



\section*{EXPERIMENT\protect}

In this section we will give our method to improve the transmission efficiency of the  traffic network, show the simulation results, and then discuss them extensively.

\noindent {\bf 1.Method}

$1)$ In order to test our method we should generate traffic- like topological networks. Here we got three kinds of networks, BA network, USAir network and California road network. To simplify the simulation, they are set to be unweighted and undirected.

$2)$ The next step is to simulate the congestion of the traffic net. We think that in the status of congestion the traffic in the route almost Stagnated, so to some degree it equals the route has been broken. So in this process we randomly select a part of the route, and break them up. Notice that we allow the situation that after the break of the network, there are exists individual nodes.

$3)$ Since we have broken links, so we can use link prediction methods to rank the edges of nonsexist, and determine the sequence of recovery links. One different from link prediction is that in link prediction we will rank all the edges of nonsexist while in route recovery we just need to judge the important of broken edges and recover them in order. Here we have eight methods to recover the routes which was mentioned before. 

$4)$To verify the efficiency of different methods based on link prediction, we have conducted package simulation experiments. Several models have been proposed to simulate packet traffic dynamics on complex networks by introducing random generation of packets in each time step and various routing strategies. We here adopt one widely used before.

{\bf In each time step}, each node generates $\frac{R}{N}$ packets, where N is the size of the network and R is a parameter tuning the generation rate, i.e., there are R packets generated in the network at each step. When $\frac{R}{N}$ is not an integer, we create Int($\frac{R}{N}$) packets determinately and create a packet simultaneously with probability p=$\frac{R}{N}$ - Int($\frac{R}{N}$), where Int($\frac{R}{N}$) is the integral part of $\frac{R}{N}$ and thus p is the fractional part. The packets are initialized with random destinations. Moreover, we set a transmission capacity $C_i$ to the node i, i=1,2, ... ,N, which means that, at each time step, the maximal number of packets transferred by the node i to the next node according to the routing table is $C_i$ . When the node cannot transfer all the packets accumulated in its queue, it deals with them following the first-in-first-out rule. Hence, the routing strategy also plays an important role in the traffic dynamics. We see that, in the Internet, the routing strategy of within domains is the shortest path algorithm and, between domains, i.e., for the AS level, the border gateway protocol causes the packets to be transmitted along almost the shortest path\cite{BVL}. Therefore, in the paper, we adopt the shortest path routing strategy. When a packet reaches its destination through the shortest path routing, the packet will be deleted from the system.In order to analyze the transition from free flow to a congested state, we use the order parameter presented in previous studies\cite{MEN},

\begin{equation}
\centerline{
$\mu<R>=$
$\lim\limits_{t \to \infty }{\frac{< \bigtriangleup w >}{R\bigtriangleup t}}$
}
\end{equation}

where W(t) is defined as the number of packets on the network at time step t, and W=W(t+t)−W(t), with $<…>$ indicating averaging over time windows of width t. In other words, the order parameterμrepresents the ratio between the existing flow and the inflow of packets calculated through a long enough time period. Obviously, in the free flow state, i.e., where there is no congestion in the network, the system can deal with the generated packets and thus the existing flow is close to zero. Otherwise, in the congested state, the number of generated packets is too large to be transmitted and the flow existing in the network will also be large, which causes the order parameter to approach 1.

\noindent {\bf 2.Metrics for evaluation}

We have got eight methods to recovery the links in the network. Different methods get different recovery efficiency. First we initialized an array like bestPerform = [0, 0, 0, 0, 0, 0, 0, 0], the array contains 8 elements and each element correspond to a method, the methods are [Random, PA, RPA, RAPA, LP, RLP, SA, RSA] in order. In a certain congestion  situation, we use these different methods to recover the links respectively, when we have finished the recovery of the roads, we will calculate the order parameter $\mu<R>$. If one method get a lowest order parameter in this period (recovery process) of time, we will choose the corresponding element in the array bestPerform and add by 1. We will repeat the process in a certain times, and finally count the elements in the  bestPerform. If an element got the highest value, then it means that the corresponding methods performs best in the link recovery.

\noindent {\bf 3.Result}

\begin{figure}[ht]
\scalebox{0.5}[0.5]{\includegraphics[trim=0 10 0 0]{road.png}}
\caption{In this experiment, we first create a California road network, the data was obtained from stanford Datasets.Intersections and endpoints are represented by nodes and the roads connecting these intersections or road endpoints are represented by undirected edges.The network contains 1,965,206	 nodes and 2,766,607 edges.Since the network is too large to process, so we need to select a part of the graph. We randomly choose a node from the network, like nodeId= 800120, then we regard the node as the center node, next we use the center node to spread out the network, just like BFS algorithm.Here in this experiment, we set the spread level equals 15, and we get the network with 223 nodes and 291 edges.}\label{BAZD}
\end{figure}

Since road network has the lowest heterogeneity among BA and USAir, the degree of the nodes in the network mainly lies in 1,2,3,4,5, so it is hard to identify the important of the edges. But from 10,000 round of comparison, our experiment still reveals the Negative Correlation effect in road network. RPA, RAPA, RLP, RSA performs better than random way, and PA, LP, SA performs badly.From the fig1, we can observe that RSA performs best, I think this may be the reason that we have fully utilize the information of one node in the situation of low heterogeneity. This phenomenon also remind us to use more similarity information to improve the performance of road recovery. 


\begin{figure}[ht]
\scalebox{0.5}[0.5]{\includegraphics[trim=0 10 0 0]{baFlow.png}}
\caption{In order to verify  the robust of different methods. We have controlled the value of R. In the above experiment, we create a USAir network, it has 332 nodes and 2126 edges. We then set the ratio of destroyed edges equals to 0.05. Under different R, we got the algorithm's performance as above.}\label{BAZD}
\end{figure}


USAir is a kind of transportation system. We can see RAPA always performs best, then follows as RLP, RSA.With the increasement of R, although negative way still performs better, but the gap between positive and negative ways are becoming smaller, the reason for this phenomenon may be that at a very small value of R, the congestion  level of the roads are not serious, so the recovery of the roads can release the congestion  level a lot. But with the increase of R and congestion, the recovery of roads can't change the congestion  situation a lot, so the difference between the recovery methods isn't as big as above. Since the performance of negative ways overwhelms positive link prediction methods under different traffic flow rate, so our methods proves to be robustness. Since normal traffic system won't be too congested, so negative methods can be far more better than random recover the routes. 


\begin{table}[tbp]
\centering  % 表居中
\begin{tabular}{lcccccccccc}  % {lccc} 表示各列元素对齐方式,left-l,right-r,center-c
\hline  % \hline 在此行下面画一横线
R &5 &10 &15 &20 &25 &30 &35 &40 &45 &50\\ 
\hline  Order parameter &0.005 &0.24 &0.34 &0.46 &0.54 &0.60 &0.64 &0.68 &0.70 &0.73\\  
\hline
\end{tabular}
\caption{Correspond to Fig.2, the average order parameter in different value of R. }
\end{table}


\begin{figure}[ht]
\scalebox{0.5}[0.5]{\includegraphics[trim=0 10 0 0]{fractionSize.png}}
\caption{In order to know how congestion  ratio affect the recovery of traffic. We have generate a BA network with 256 nodes and 508 edges, x-axis represent the ratio of congested route. In the above picture, 0.04 means there are int(0.04*508) edges have been congested. Y-axis represent the accurate prediction sequences in 1000 round experiment. The packages generated in one time step is 10.}\label{BAZD}
\end{figure}


A lot of traffic system can be modeled as BA network. From fig.3 we can see with the increasement of fraction ratio, negative recovery methods perform far more better than random method and positive one's. At the fraction ratio of 0.4, in 1000 round experiment, RSA method performs best in about 600 round comparison, RPA and RAPA performs about 200 round best respectively while other methods almost have no chance in the guidance of traffic recovery. I think there are two reason for the result, with the increasement of fraction ratio, the recovery sequences have  more combination so we can differentiate them more easily. Another reason is that big fraction ratio requires more time in the recovery process, in a larger period of time, we can verify our methods more easily. 


\begin{figure}[ht]
\scalebox{0.5}[0.5]{\includegraphics[trim=0 10 0 0]{baSize.png}}
\caption{To test our methods in different size of networks, we have generate five BA networks, they contain nodes with 100, 200, 300, 400, 500 respectively. The probability for each node to generate a package in one time step is 0.02 and the fraction ratio is 0.05.}\label{BAZD}
\end{figure}


In fig.4 we can see that when the network contains 100 nodes, eight methods haven't much difference in the traffic recovery, this is mainly because that at a very small network there are no big difference for each node so we can't distinguish them clearly. But when the size of the net reaches a certain scale, nodes in the network play roles quite differently, hub nodes and normal nodes can't be the same. When the size of BA network reaches 200, the difference of performance can be observed easily. With the increasement of BA size, negative recovery methods perform better than random method and positive methods. At the size of 500, positive ways almost have no chance in accurately recover the routes in a certain sequence. The reason for that negative methods in larger BA network plays better mainly because the larger BA network has larger heterogeneity, so the nodes play roles differently and we can distinguish them more easily. Normally traffic system won't be too small, they usually contain nodes over 200, so our negative ways can be of use in routes recovery especially the reverse similarity(RSA) method. 


\section*{CONCLUSION}
Traffic  problem affect the operation efficiency of the society, current study mainly focus on add new route or rebuild the network. But even the best system can't avoid congestion  entirely. Our methods mainly focus on how to recover the routes sequences in a certain order to minimize the negative impact. Inspired by link prediction methods, we have recovered the the route in three strategies. These strategies are positive ways which include PA, LP and SA; negative ways which include RPA, RAPA, RLP and RSA; random way. From our experiment we find that negative ways perform better than the random recover of routes, random way are better than positive ways which is obviously since negative ways and positive ways are reverse methods. Beside the verification of negative link prediction methods in routes recovery. We also find that larger network can improve negative link prediction methods' accuracy; Higher degree heterogeneity improves experiment accuracy also, so negative methods are effective especially for BA like topological; At a high level of congested routes like half of the routes have been broken, negative methods almost have accuracy of 100\% in the routes recovery.

Our study is tentative in using link prediction methods solving traffic problem, our result shows a tremendous potential combine link prediction methodology and real traffic problem. Since few people have doing researches in this field, so there are many shortcomings to be further optimized. In lack of real traffic flow data, so we simplified our simulation with a certain rate of packages generated in a certain time step. If possible, further studies can obtain real traffic flow and doing recovery experiment. Also weighted networks deserve to be discussed. In all, our research have found some simple ways to recover the congested ways efficiently, this may a new branch in the traffic problem.

\section*{Author contributions}
Yichao Zhang and Xing Li contribute equally.







\section*{References}
\begin{thebibliography}{00}

\bibitem{TFWN}Traffic fluctuation on weighted networks

\bibitem{PSV} R. Pastor-Satorras and A. Vespignani, “Epidemic spreading in scale-free networks,” Phys. Rev. Lett., vol. 86, pp. 3200–3203, 2001.

\bibitem{AEM} A. E. Motter and Y. C. Lai, “Cascade-based attacks on complex net- works,” Phys. Rev. E, vol. 66, no. 065102, 2002.

\bibitem{KIG} K. I. Goh, D. S. Lee, B. Kahng, and D. Kim, “Sandpile on scale-free net- works,” Phys. Rev. Lett., vol. 91, no. 148701, 2003.

\bibitem{BTS} B. Tadic ́, S. Thurner, and G. J. Rodgers, “Traffic on complex networks: Towards understanding global statistical properties from microscopic density fluctuations,” Phys. Rev. E, vol. 69, no. 036102, 2004.

\bibitem{LZYC} L. Zhao, Y. C. Lai, K. Park, and N. Ye, “Onset of traffic congestion in com- plex networks,” Phys. Rev. E, vol. 71, no. 026125, 2005.

\bibitem{HHM} H. Hong, M. Y. Choi, and B. J. Kim, “Synchronization on small-world net- works,” Phys. Rev. E, vol. 65, no. 026139, 2002.

\bibitem{RAA} R. Albert and A.-L. Barabási, “Statistical mechanics of complex net- works,” Rev. Mod. Phys., vol. 74, pp. 47–97, 2002.

\bibitem{SND} S. N. Dorogvtsev and J. F. F. Mendes, “Evolution of networks,” Adv. Phys., vol. 51, pp. 1079–1187, 2002.

\bibitem{MEJ} M. E. J. Newman, “The structure and function of complex networks,” SIAM Rev., vol. 45, pp. 167–256, 2003.

\bibitem{SBV} S. Boccaletti, V. Latora, Y. Moreno, M. Chavez, and D.-U. Hwanga, “Com- plex networks: Structure and dynamics,” Phys. Rep., vol. 424, pp. 175–308, 2006.

\bibitem{TCIC}Traffic congestion in interconnected complex networks

\bibitem{GTC}N. Biggs, E. K. Lloyd, and R. J. Wilson. Graph Theory, 1736-1936. Claren- don Press, New York, NY, USA, 1986.

\bibitem{LPU}MohammadAlHasan,VineetChaoji,SaeedSalem,andMohammedZaki. Link prediction using supervised learning. In In Proc. of SDM 06 workshop on Link Analysis, Counterterrorism and Security, 2006.

\bibitem{NPA}Ryan N. Lichtenwalter, Jake T. Lussier, and Nitesh V. Chawla. New per- spectives and methods in link prediction. In Proceedings of the 16th ACM SIGKDD International Conference on Knowledge Discovery and Data Min- ing, KDD ’10, pages 243–252, New York, NY, USA, 2010. ACM.

\bibitem{LPC}Zhen Liu, Qian-Ming Zhang, Linyuan Lü, and Tao Zhou. Link prediction in complex networks: A local naïve bayes model. EPL (Europhysics Letters), 96(4):48007, 2011.

\bibitem{WWC}Andrew Chen-Brian Tran Ole J. Mengshoel, Raj Desai. Will we connect again? machine learning for link prediction in mobile social networks. 2013.

\bibitem{SRM}Kai Yu, Wei Chu, Shipeng Yu, Volker Tresp, and Zhao Xu. Stochastic
relational models for discriminative link prediction. In Advances in Neural Information Processing Systems, pages 333–340. MIT Press, 2007.

\bibitem{GME}Edward J Taaffe,Howard L Gauthier,Morton E.0’KeHy.
Geography of Transportation(2nd).Prentice Hall,1996.

\bibitem{GOT}odrigue J P et a1.ne Geography of Transport Systems,
Hofstra Univemity.Department of Economics\& Geogra

\bibitem{UTN}Sheffi,Yose[Urban Transportation Networks:Equilibrium Analysis with Mathematical Programming Methods.Prentice-Hall,Inc,1 985.

\bibitem{NHS}A. Clauset, C. Moore, M. E. J. Newman, Hierarchical structure and the prediction of missing links in networks, Nature 453 (2008) 98.

\bibitem{TOM}S. Redner, Teasing out the missing links, Nature 453 (2008) 47.

\bibitem{SPR}J. O’Madadhain, J. Hutchins, P. Smyth, Prediction and ranking algorithms for event-based network data, In Proceedings of SIGKDD 2005, ACM Press, New York, 2005, p. 23.

\bibitem{TZ}T. Zhou, J. Ren, M. Medo, Y.-C. Zhang, Bipartite network projection and personal recommendation, Phys. Rev.E 76 (2007) 046115.

\bibitem{SAD}T. Zhou, Z. Kuscsik, J.-G. Liu, M. Medo, J. R. Wakeling, Y.-C. Zhang, Solving the apparent diversity-accuracy dilemma of recommender systems, Proc. Natl. Acad. Sci. U.S.A. 107 (2010) 4511.

\bibitem{CAA}W. Zeng, M.-S. Shang, Q.-M. Zhang, L. Lu ̈, T. Zhou, Can dissimilar users contribute to accuracy and diversity of personalized recommendation, Int. J. Mod. Phys. C (to be published).

\bibitem{SV}Lü, Linyuan, and T. Zhou. "Link prediction in complex networks: A survey." Physica A Statistical Mechanics \& Its Applications 390.6(2011):1150-1170.


\bibitem{WG}Xie Y B, Zhou T, Wang B H. Scale-free networks without growth[J]. Physica A Statistical Mechanics \& Its Applications, 2008, 387(7):1683-1688.

\bibitem{LI}Zhou T, Lü L, Zhang Y C. Predicting missing links via local information[J]. \bibitem{AC}The European Physical Journal B, 2009, 71(4):623-630.
Ackland R, Ackland R. Mapping the U.S. Political Blogosphere: Are Conservative Bloggers More Prominent?[C]// 2005.

\bibitem{BA}A.-L. Barabási and R. Albert, Science 286, 509 (1999).

\bibitem{PML}Zhou T, Lü L, Zhang Y C. Predicting missing links via local information[J]. The European Physical Journal B, 2009, 71(4):623-630.
71(4):623–630. 

\bibitem{KD}Zhang Y, Aziz-Alaoui M A, Bertelle C, et al. Knowledge Diffusion in Complex Networks[C]// IEEE, Intl Conf on Ubiquitous Intelligence and Computing and 2015 IEEE, Intl Conf on Autonomic and Trusted Computing and 2015 IEEE, Intl Conf on Scalable Computing and Communications and ITS Associated Workshops. 2015.

\bibitem{SW}Watts DJ, Strogatz SH. Collectivedynamics of ’small-world’ networks[C]// Nature. 1998:440-442.

\bibitem{BVL}S. Boccaletti, V. Latora, Y. Moreno, M. Chavez, and D.-U.Hwang, Phys. Rep. 424, 175 (2006).

\bibitem{MEN}S. N. Dorogovtsev, J. F. F. Mendes. Evolution of networks[M]// Evolution of networks :. Oxford University Press, 2003:1842-1845.


\bibitem{LCS}[1] L. Cui, S. Kumara, and R. Albert, IEEE Circuits and Systems Magazine 10, 10 (2010).
\bibitem{LZY}[2] L. Zhao, Y.-C. Lai, K. Park, and N. Ye, Phys. Rev. E 71, 026125 (2005).
\bibitem{AAA}[3] A. Arenas, A. D ́ıaz-Guilera, and R. Guimer`a, Phys. Rev. Lett. 86, 3196 (2001).
\bibitem{DDL}[4] D. De Martino, L. Dall’Asta, G. Bianconi, and M. Mar- sili, Physical Review E 79, 015101 (2009).
\bibitem{JWC}[5] J. Wu, C. Tse, F. C. Lau, and I. W. Ho, IEEE Trans. Circuits Syst. I, Reg. Papers 60, 3303 (2013).
\bibitem{SBVV}[6] S. Boccaletti, V. Latora, Y. Moreno, M. Chavez, and D.- U. Hwang, Phys. Rep. 424, 175 (2006).
\bibitem{RGA}[7] R. Guimer`a, A. Arenas, A. D ́ıaz-Guilera, and F. Giralt, Phys. Rev. E 66, 026704 (2002).

\bibitem{SCW}[9] S. Chen, W. Huang, C. Cattani, and G. Altieri, Mathe- matical Problems in Engineering 2012, 732698 (2012).
\bibitem{GYT}[10] G. Yan, T. Zhou, B. Hu, Z.-Q. Fu, and B.-H. Wang, Phys. Rev. E 73, 046108 (2006).
\bibitem{BDY}[11] B. Danila, Y. Yu, J. A. Marsh, and K. E. Bassler, Phys. Rev. E 74, 046106 (2006).
\bibitem{XLM}[12] X. Ling, M.-B. Hu, R. Jiang, and Q.-S. Wu, Phys. Rev. E 81, 016113 (2010).
\bibitem{FTY}[13] F. Tan and Y. Xia, Physica A 392, 4146 (2013).
\bibitem{YXD}[14] Y. Xia and D. Hill, EPL 89, 58004 (2010).
\bibitem{XLMD}[15] X. Ling, M.-B. Hu, J.-C. Long, J.-X. Ding, and Q. Shi,
Chin. Phys. B 22, 018904 (2013).
\bibitem{GQZ}[16] G.-Q. Zhang, S. Zhou, D. Wang, G. Yan, and G.-Q.
Zhang, Physica A 390, 387 (2011).
\bibitem{ZTKE}[137] Z. Toroczkai, K.E. Bassler, Nature 428 (2004) 716.
\bibitem{ZTBE}[138] Z. Toroczkai, B. Kozma, K.E. Bassler, N.W. Hengartner, G. Korniss, preprint cond-mat/0408262.
\bibitem{TORS}T. Ohira, R. Sawatari, Phase transition in a computer network traffic model, Phys. Rev. E 58 (1998) 193–195.

\end{thebibliography}



\end{document}
    